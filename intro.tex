\section{Introduction}
In this project we have designed an application which fullfils two tasks concerning the use of rooms at the IT University (ITU). We have developed a user friendly interface for managing the meeting and classrooms. This has been split into two distinct parts. The ad-hoc handling of meeting and classrooms (\textbf{day to day booking}), and an interface which helps the staff in assigning rooms to courses at the start of a semester (\textbf{semester planning}).\\

Currently, no application keeps track of the day to day bookings, and it is instead handled manually by pen and paper.\\
Our day to day booking interface will provide a larger degree of flexibility when finding a place to work, conduct meetings etc., as it provides the possibility to book rooms in advance and get an overview of the currently available rooms.\\

Additionally, the current semester planning is inefficient, decentralized and very time consuming. This process is also done manually, though with some support from an application (Tabulex\cite{tabulex}) normally used in elementary schools.
We want to construct a system that centralizes the semester planning, and aids the persons responsible as much as possible, by making it easier for the staff to make a decision based on the information and suggestions provided by the application. \\

The program has been implemented as a web based solution, to make it globally accessible. We have chosen a web framework named Lift, written in Scala, which provides us with some needed functionality out of the box. This allows us to put our main focus on creating a solid user interface. Due to this choice of focus, some parts of the system have not been implemented.


%-- This goes in current situation instead!
%When two courses overlap each other, the program does not suggest any alternative solutions to this conflict. This makes it difficult, and very time consuming to come up with a proper alternative solution, as one change often results in another conflict.\\
%Finally, Tabulex does not keep track of the capacity of each rooms, which makes it impossible to do a proper allocation, as you cannot supply the program with knowledge about how the rooms are actually used.\\