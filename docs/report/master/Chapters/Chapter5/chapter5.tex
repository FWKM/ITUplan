\chapter{Evaluation}
\label{chap:evaluation}

\section{Impact}
\label{sec:impact}
The following is an attempt to estimate what impact a fully implemented version of our system would have for the relevant stakeholders.

\subsection{Launch}
Before a fully implemented system could be launched, policies and decisions should be made for the issues raised in section \ref{sec:domain_challenges}.
With any kind of changes to current procedures in an institution of this size, there will always be a transistion period where things are not being used properly, not being used at all. We hope that the intuitive and user friendly approach we have taken, will help ease this particular part of the launch.

\subsection{Facilities Management}
As previously stated, FM handles the day to day booking manually, by manning a desk and browsing/updating a binder.
With our system it will depends on whether or not they choose the approval approach mentioned in section \ref{subsec:approval}.
In case of using approval, their work would be faster, but in the same approximate amount.
If they instead choose to follow our recommendations and allow users to make their own bookings, in limited number, and have the power to delete them, their time managing rooms could be greatly reduced.

\subsection{Study Administration}
This is the single greatest benefactor of a succesful implementation, as the current semester is highly time consuming, and quite troublesome. With our system, the need for tabulex would be gone, along with the manual work to fill the gaps that tabulex cannot do automatically. Through our discussion with the representative from the Study Administration, she mentioned that getting the overview of whether a not a track was fully assigned was the real problem. Our proposed design suggest a track-based approach with easy overview on how the schedule and assignments for a single track looks. The automated suggestions for clash-handling would also enable them to make educated decisions on what courses need to be moved and which rooms to be reassigned. Naturally, our system would not be able to solve all the problems encountered in such a complex assignment, but \emph{any} suggestion is more than what they have currently.

\subsection{Students, faculty and staff}
For the daily visitors to ITU, finding a meeting room or workspace would be greatly simplified, as the need to wait in line at the information desk would be eliminated. Additionally, rooms can be reserved from home, so no one would have the dissappointment of showing up, only to discover that no room is available.

\section{Possible improvements}
\todo{If this was a commercial product, what would we change/add/fix etc.}

\section{Expansions}
\todo{Consider if this section is needed. Perhaps it could fit into the above.
If not, we could talk about ridding the need for tabulex, centralizing everything in our system and so forth.}

\section{Reflection}
\label{sec:reflection}
In this chapter, we will evaluate the process of the project as a whole, to reflect on the actual success of the choices we made througout the project.

\subsection{Lift}
We chose Lift as our framework of choice, because of the rich features it contained, which could serve our project well. Before the project, we had no experience with Lift or Scala, but Lift seemed like a good choice for our application, and while we have lots of experience with java, we thought that Lift would be easy to get used to.\\
However, we quickly found out that Lift was a big mouthful. Lift uses the functional paradigms extensively, and combined with the lack of documentation, it was very hard for us to figure out what the various classes was for, and how they should be used. We had a very hard time writing even simple test applications.\\
We did however manage to use the features of Lift that we intended, but the problems we had with Lift are reflected in the final result of our product. We did not have time to implement as much of the application as we wanted, and we do not use the full potential of Lift in some of our classes. \\

When looking back, Lift was still the right choice for the application, but probably not the right choice for us personally. We would have been much more effective if we used something we were familiar with, like PHP, from the beginning. However, we do not regret our choice, as the experience we have had with Lift is valuable, if we at some point in the future have to use Lift again.

\subsection{Choice of project}
This type of project has been very different to us, as none of our previous projects, or courses for that matter, have been focusing on interfaces. We have only had one single course about designing user interfaces, on our first semester. \\
This project was a nice change of phase, and something we enjoyed and learned a lot from. During the project, we learned how much of a difference a well designed user interface actually makes, when doing the usability tests.\\
While it has been a good experience, it has also been a challenge. Becuase of our rather limited knowledge and experience with GUI building, we basically started from scratch. We had to study a lot of litterature to be able to make base our choices on other than personal taste and intuition. Before we finally decided on the first mockup, which we afterwards tested and changed, we spent a lot of time on initial mockups written on whiteboards, which later were discarded because they simply did not work the next day we revisited them. This process was repeated 4 times before the first initial mockup of the frontpage, and a couple of times in between the iterations, after the actual usabilty tests.\\ With all initial and tested mockups included, we've done about 15 mockups for our relative simple application.