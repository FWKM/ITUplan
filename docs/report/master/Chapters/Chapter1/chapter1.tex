\chapter{Background}
\section{Introduction}
In this project we have designed an application which fullfils two tasks concerning the use of rooms at the IT University (ITU). We have developed a user friendly interface for booking of rooms(\textbf{day to day booking}), and an interface which helps the staff in assigning rooms to courses at the start of a semester (\textbf{semester planning}).\\

Currently, no application keeps track of the day to day bookings. This is done manually by the Facility Management(FM).\\
Our day to day booking interface will provide a larger degree of flexibility when finding a place to work, conduct meetings etc., as it provides the possibility to book rooms in advance and get an overview of the available rooms at the moment.\\

 Additionally, the current semester planning is inefficient, decentralized and very time consuming. This process is currently done manually, with support from an application (Tabulex\cite{tabulex}) normally used in elementary schools. When two courses overlap each other, the program does not suggest any alternative solutions to this conflict. This makes it difficult, and very time consuming to come up with a proper alternative solution, as one change often results in another conflict.\\
Finally, Tabulex does not keep track of the capacity of each rooms, which makes it impossible to do a proper allocation, as you cannot supply the program with knowledge about how the rooms are actually used.\\
We want to construct a program that centralizes the semester planning, and aids the persons responsible as much as possible, by making it easier for the staff to make a decision based on the information and suggestions provided by the application. \\

The program has been implemented as a web based solution, to make it globally accessible. As our main focus is to create a solid user interface, parts of the application will not be implemented. With this in mind, we have chosen a web framework named Lift, written in Scala, which provides us with some needed functionality out of the box.

\pagebreak
\label{sec:current_situation}
\section{Current situation}
As of writing, room management at the ITU is split between the Study Administration and the Facilities Management. Within those departments there is another division of labour which all in all adds to the confusion of handling the limited rooms efficiently.

\subsection{Facilities Management}
The information desk, controlled by FM, is currently in charge of the day to day use of the meeting rooms. When a person wishes to reserve or use a room, the FM employee will refer to a binder with a paper-overview of the current status of each single room.\\

This approach presents several problems:
\begin{itemize}
	\item It is not possible for the students to book a room in advance. Employees can book rooms in advance for conferences, faculty meethings etc., but the students can only book a room on the current day.
	\item It is not possible to get an overview of the rooms currently available. This is a problem when you have specific requirements like the need for specific equipment, or when you need a room with a specific capacity. FM can answer some of these inqueries, but they rely on experience as they do not have this information available anywhere.\\ Additionally, only FM can tell you if and which rooms are currently unbooked, which at times can be a time consuming process during busy periods.
\end{itemize}

\subsection{Study Administration}
In regards to room management, the Study Administrations biggest challenge is the planning of a semester. First step is to collect data from the course database to see which courses will be held in the coming semester. When a list of courses have been compiled, a system called Tabulex is used to create a draft schedule to avoid overlaps of teachers and courses required by several different educations. The output from Tabulex is finally used to manually create the room assignments for a standard week of the semester.

The other function of the Study Administration is to assign rooms required by exams, for both the exam itself and for the prepation-time required by some exams. Unfortunately, not all courses have ended by the time exams begin, which further limits the available rooms.\\

\label{sec:scope}
\section{Scope}
 Since we want to focus on the creation of a solid interface that solves the above problems, the implementation has been a second priority. We have divided the project into two main tasks, which together constitute to the main body of such an application. The success of the project depends on how well the interfaces for these two tasks have been designed.\\
In the following, we will describe the two main tasks and how they relate to each other.

\subsection{Day to day booking}
Our program should provide an interface to support booking of rooms at the IT University. Users of this function might have very different requirements. Students might just need a room to work, while an employee might book a specific room in advance for a meeting. Others might need specific equipment to be available. Because of the many requirements, the program should not only provide the user with options, but also suggest solutions.\\
Since most people using our application will be students looking for a room without any specific requirements, the simple task of finding any available room should be easy accessible and quick to use. \\

The day to day booking should thus contain:

\begin{itemize}
	\item A general overview of the available rooms to book.
	\item An interface that quickly books a room, using a minimum of options. (\textbf{Quick booking})
	\item An interface that presents the user with several options, such as the need for specific equipment, room capacity, time of day, etc. (\textbf{Advanced booking})
\end{itemize}

The application should:

\begin{itemize}
	\item Suggest \emph{a single room} to the user, when using the quickbook function. The room suggested, should be the \emph{optimal} room, in terms of efficient room allocation. This means that the application should suggest the smallest room, with the least amount of equipment available.
	\item Suggest a list of rooms that fulfil the maximum number of requirements, given as input by the user through the selective booking function. If all requirements can be fulfilled, only the optimal rooms (see above) should be presented to the user.
\end{itemize}

To exemplify a working application, the day to day booking will be fully implemented.

\subsection{Semester planning}
In addition to the day to day booking, we will design the user interface for a semester planning system. Planning a semester is in the case of ITU not just a matter of assignment. Each semester have more courses and day/time combinations than there is rooms, so the essence of it,is providing the tools to prioritize in an informed manner. At the point when a semester needs to be planned, most courses only have a start-week and an end-week, along with the number of enrolled students. Some courses can, eg. due to external teachers with busy schedules, have narrowed down their options, and we must of course including support for these scenarios.
Worth noting, and part of our considerations in the design, is that there will only be a few people ever using this part of the system, and these people will be easily accessible for training.

The semester planning should contain:

\begin{itemize}
	\item An interface to manage courses
	\item An interface to manage the aforementioned tracks
	\item An interface to handle any problems with planning that cannot be resolved automatically
	\item An interface to view/review the current semester's allocation
\end{itemize}

The application should:

\begin{itemize}
	\item Attempt to not put courses in meeting rooms, unless all class rooms are unavailable in all possible scenarios
	\item Provide alternatives to assignments and allow for wishes/preferences
	\item Provide suggestions for resolutions if no full semester plan is possible
\end{itemize}


We will design the user interface for the semester planning, but not implement or add the required functionality. Doing the full scale implementation would be a possibility, but it is not crucial to prove our point of data presentation and usability. We have therefore chosen to perform the analytical work, and leave out large parts of the actual implementation.

\subsection{Relations}
Even though the two tasks present very different functionality, the relationship between them needs to be defined, as they both occupy rooms. \\

The semester planning should have precedence at all times. This means that any day to day bookings interfering with the semester planning will be deleted. Because of this, it should also not be possible to book a room during the use of the semester planning interface. \\
A day to day booking cannot be made, if it overlaps with a scheduled event (like a lecture) created through the event planning interface.\\

Not all users should have access to the semester planning. Only users defined as \emph{administrators} have access to this part of the system.

\subsection{Functionality out of scope}
Some functionality are not relevant to the functionality and interface of the two main tasks in our scope. The following list presents components that we discuss in this report, but do not implement.

\begin{itemize}
	\item A personal interface, listing all booking made by the user. Through this interface, it should be possible to review, edit and delete bookings. Each administrator of the system, should have access to \emph{all} booking, and should be able to approve or disapprove bookings.
	\item A communication protocol for sending and receiving messages. This is nessesary if e.g. a booking is deleted by an administrator through the semester planning interface.
	\item Some kind of restrictions to the day to day booking. It should e.g. not be possible for a user to book all available rooms at the same time.
\end{itemize}


\section{Target audience} % (fold)
\label{sec:target_audience}
Our application is implemented solely for ITU. It would have been possible to implement a general solution capable of adapting to many instutions, and while this is our general line of thought, a good UI needs to contain various elements specific to the domain in which it is used. This includes a map of the university for easy navigation and localization of rooms, and many application specific options. These features might not be nessesary for e.g. another university, which examplifies the difficulty in creating a general room-booking application without compromising the required functionality.\\

Our target audience can thus be divided into two groups:
\begin{itemize}
\item Student body at ITU
\item Employees at ITU
\end{itemize}

\subsection*{Student body}
The student body covers ages 18 - 40+ and with very different backgrounds. Based on the nature of the university, it is fair to assume that most students have atleast basic familiarity with IT and computer use in general.

\subsection*{Employees}
The computer proficiency of the employees can range from expert to novice, and no clear-cut distinctions can be made. We have, however, observed that the majority of the people our system would affect are using tools with some similarities, and are therefore assuming their proficiency to be above novice users.\\

Note that even though our day to day booking interface will mostly be used by students, we can make no assumptions that the target audience for this part of the application \emph{only} contains students. We need to make sure the day to day booking can be used by a very wide range of people, including also the employees.
