\chapter{Background}
\section{Introduction}
In this project we have designed a user-friendly room booking experience for the IT University\footnote{IT University will henceforth be refered to as 'ITU'}, and provided the concept for a solution to the complex task of assigning rooms to courses. The current process of assigning rooms to both students as well as courses, is both tedious, inefficient and without proper support. At the moment, it is hard to get an overview of how rooms are used and how efficient the current allocation is. This makes it hard for the staff to exploit the full capacity of the university, which is of high importance to expecially ITU, because of the rather limited number of rooms and auditoriums. \\

We want to construct a program that centralizes this process, and aids the persons responsible as much as possible. Such a system would make the day to day task of booking a room for group work, seen from the perspective of the students, much more flexible. The harder to grasp tasks such as completing the room assignments for a whole semester should be supported by the system, to make it easier for the employees to make the right decisions.\\ As it should be possible to book a room at home, on the way to school or on site, we have implemented the system as a web-based solution. As our main focus is to create a solid user interface, parts of the application will not be implemented. With this in mind, we choose lift as our webframework, as it provides many features relevant to our product out of the box. 

\pagebreak
\label{sec:current_situation}
\section{Current situation}
As of writing, room management at the ITU is split between the Study Administration and the Facilities Management. Within those departments there is another division of labour which all in all adds to the confusion of handling the limited rooms efficiently.

\subsection{Facilities Management}
The information desk, controlled by FM, is currently in charge of the day-to-day use of the meeting rooms. When a student or teacher wishes to reserve or use a room, the FM employee will refer to a binder with a paper-overview of current status, and if they succesfully find an available room a form is filled out, photocopied and archived.

Aside from this, another part of FM is in charge of assigning offices and workspaces to teachers, teaching assistants and researchers. This is done through a locally stored Microsoft Access database, and is handled primarily by one person.

\subsection{Study Administration}
In regards to room management, the Study Administrations biggest challenge is the planning of a semester. First step is collect data from the course database to see which courses will be held in the coming semester. When a list of courses have been compiled, a system called Tabulex\footnote{http://www.tabulex.dk/} is used to create a draft schedule so no overlaps of teachers and courses required by several tracks\footnote{'Track' is the term used to describe a semesters worth of courses in a specific direction, eg. the 'Software Engineering' direction of the 'Software Development and Technology' MSc education}. The output from Tabulex is finally used to manually create the room assignments for a standard week of the semester.

The other function of the Study Administration is to assign rooms required by exams, for both the examnation itself and for the prepation-time required by some exams. Unfortunately, not all courses have ended by the time exams begin, so apart from rooms being occupied the problem of privacy and noise presents itself.
\\
\todo{Perhaps reformatting and elobaration }
\label{sec:scope}
\section{Scope}
Our approach to designing this system, has been primarily focused on usability and all that entails. As described in \ref{sec:current_situation}, the current situation holds multiple challenges. Tedious and de-centralized management of a single resource is both time consuming and highly error-prone.\\
We want to implement a system which fulfills two main tasks:
\begin{itemize}
	\item 1. Helps the student to find a place to work
	\item 2. Aids the staff in the allocation of rooms
\end{itemize}
A large part of fulfilling these tasks is the creation of an interface, which by itself maintains a high level of support. 
% Uddyb
Both of the above tasks can often be solved in multiple ways. A student query the database for a room with specific requirements, might lower his requirements if there is no results. On the other hand, he might decide to postphone  the activity and try again when the specific room is available.\\
In comparison, the room allocation problem can contain not only a few solutions (which a program might be able to suggest), but hundreds of ways to allocate courses. When something is impossible to solve, the user has to be able to tell the computer what to prioritize, or which solution to use.
A well thought out application succeeds in presenting this information to the user, and gives a proper overview so the best decisions can be taken. \\
Therefore, the far most important and interesting aspect of our application is the userinterface, in which we have choosen to put our focus. To exemplify a solid solution to this problem, and to keep the interface focused, we have narrowed our vision to the most used area: The day-to-day booking, and the semester planning (hence the tasks mentioned above). \\

The simple booking have been succesfully prototyped, whereas the semester planning have only been designed and developed theoretically, and not implemented. Doing the full scale implementation of the semester planning would be a possibility, but it is not crucial to prove our point of data presentation and usability. We have therefore chosen to perform the analytical work, and leave out large parts of the actual implementation.

\todo{Limitations/inclusions of our project work goes here}
