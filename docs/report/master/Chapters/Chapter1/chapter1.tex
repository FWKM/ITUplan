\chapter{Background}
\section{Introduction}
In this project we have designed a user-friendly room booking experience for the IT University\footnote{IT University will henceforth be refered to as 'ITU'}, and provided the concept for a solution to the complex task of assigning rooms to courses. As described later, the current process in which all room assignments - both short and long term - is both tedious and inefficient. With the ever increasing focus on web-based technology, we found it only logical to bring this part of university life online in a user-friendly manner by providing a simple overview for both day-to-day tasks such as reserving a room for group work, and the larger, harder to grasp tasks such as completing the room assignment for a whole semester.

\pagebreak
\section{Current situation}
As of writing, room management at the ITU is split between the Study Administration and the Facilities Management. Within those departments there is another division of labour which all in all adds to the confusion of handling the limited rooms efficiently.

\subsection{Facilities Management}
The information desk, controlled by FM, is currently in charge of the day-to-day use of the meeting rooms. When a student or teacher wishes to reserve or use a room, the FM employee will refer to a binder with a paper-overview of current status, and if they succesfully find an available room a form is filled out, photocopied and archived.

Aside from this, another part of FM is in charge of assigning offices and workspaces to teachers, teaching assistants and researchers. This is done through a locally stored Microsoft Access database, and is handled primarily by one person.

\subsection{Study Administration}
In regards to room management, the Study Administrations biggest challenge is the planning of a semester. First step is collect data from the course database to see which courses will be held in the coming semester. When a list of courses have been compiled, a system called Tabulex\footnote{http://www.tabulex.dk/} is used to create a draft schedule so no overlaps of teachers and courses required by several tracks\footnote{'Track' is the term used to describe a semesters worth of courses in a specific direction, eg. the 'Software Engineering' direction of the 'Software Development and Technology' MSc education}. The output from Tabulex is finally used to manually create the room assignments for a standard week of the semester.

The other function of the Study Administration is to assign rooms required by exams, for both the examnation itself and for the prepation-time required by some exams. Unfortunately, not all courses have ended by the time exams begin, so apart from rooms being occupied the problem of privacy and noise presents itself.
\\
\todo{Perhaps reformatting and elobaration }
\section{Scope}
Our approach to designing this system has been primarily focused on usability and all that entails. Tedious and de-centralized management of a single resource is both time consuming and highly error-prone.
To exemplify a solid solution to this problem, we have narrowed our vision to the most used area (day-to-day booking) and the most tedious area (semester planning).

The simple booking have been succesfully prototyped, whereas the semester planning have only been designed and developed theoretically, and not implemented. Doing the full scale implementation of the semester planning would of course be a possibility, but it is not crucial to prove our point of data presentation and usability and we have therefore chosen to perform all the analytical work on it instead.
\\ \\
\noindent As mentioned in the original problem definition, we have also elected not to include an administration panel for super-users, as it would serve little purpose without all the features of the full system.

\todo{Limitations/inclusions of our project work goes here}
