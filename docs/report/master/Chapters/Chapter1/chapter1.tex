\chapter{Background}
\section{Introduction}
In this project we have designed an application which fullfils two tasks concerning the use of rooms at the IT University (\emph{ITU}). We have developed a user friendly interface for managing the meeting and classrooms. This has been split into two distinct parts. The ad-hoc handling of meeting and classrooms (\emph{day to day booking}), and an interface which helps the staff in assigning rooms to courses at the start of a semester (\emph{semester planning}).\\

Currently, no application keeps track of the day to day bookings, and it is instead handled manually by pen and paper.\\
Our day to day booking interface will provide a larger degree of flexibility when finding a place to work, conduct meetings etc., as it provides the possibility to book rooms in advance and get an overview of the currently available rooms.\\

Additionally, the current semester planning is inefficient, decentralized and very time consuming. This process is also done manually, though with some support from an application (Tabulex\cite{tabulex}) normally used in elementary schools.
We want to construct a system that centralizes the semester planning, and aids the persons responsible as much as possible, by making it easier for the staff to make a decision based on the information and suggestions provided by the application. \\

The program has been implemented as a web based solution, to make it globally accessible. We have chosen a web framework named Lift, written in Scala, which provides us with some needed functionality out of the box. This allows us to put our main focus on creating a solid user interface. Due to this choice of focus, some parts of the system have not been implemented.\\

Section xx presents a glossary explaining the various terms and definitions used throughout the report.

\pagebreak
\section{Domain}
\label{sec:domain}
This sections contains information about the actual domain in question; the IT University in Copenhagen, and the various resources and people relevant to our application. \\
ITU has 4 different room types, which is managed by two different departments.

\begin{itemize}
	\item \textbf{Meeting rooms}\\
	Used for small to medium sized meetings, student group work and exam preparation. 
	\item \textbf{Class rooms}\\
	Mainly used for lectures, exercises and examinations. Also used for events and meetings, if none of the meeting rooms are available or large enough. Class rooms also include the auditoriums.
	\item \textbf{Offices}\\
	Used by the professors, staff members and ph.d. students.
\end{itemize}

As of writing, room management is split between the Study Administration (SA) and the Facility Management (FM). Within the study administration, there is another division of labour which all in all adds to the confusion of handling the limited rooms efficiently.

\subsection{Facilities Management}
FM manages daily adhoc planning, event planning and a subset of the rooms; the office rooms and the current day to day booking: When a person wishes to reserve or use a meeting room, the FM employee will refer to a binder with a paper-overview of the current status of each single room.\\

This approach presents several problems:
\begin{itemize}
	\item It is not possible for the students to book a room in advance. Employees can book rooms in advance for conferences, faculty meethings etc., but the students can only book a room on the current day.
	\item It is not possible to get an overview of the rooms currently available. This is a problem when you have specific requirements like the need for specific equipment, or when you need a room with a specific capacity. FM can answer some of these inqueries, but they rely on experience as they do not have this information available anywhere.\\ Additionally, only FM can tell you if and which rooms are currently unbooked, which at times can be a time consuming process during busy periods.
\end{itemize}

\subsection{Study Administration}
The SA manages the class rooms and the auditoriums, thus also in charge of allocating rooms for lectures and examinations. \\
In regards to room management, the Study Administrations biggest challenge is the planning of a semester. First step is to collect data from the course database to see which courses will be held in the coming semester. When a list of courses has been compiled, a system called Tabulex is used to create a draft schedule to avoid overlaps of teachers and courses. On the computers at ITU, it takes Tabulex up to 8 hours to generate this data. The output from Tabulex is finally used to manually create the room assignments for a standard week of the semester.\\
Note that a semester contains several different classes and study programmes. The SA uses a term called tracks to divide the various study programmes into sections. A track is one specific study programme at one specific semester, e.g. \emph{Software Development 3. Semester}. Each track has a weekly schedule, which is created manually in the final process of the semester planning.

\subsection{Current systems}
\label{sub:current_systems}
\subsubsection*{Excel sheets and Word documents}
The Study Administration uses an Excel sheet to get an overview of the availability of the different rooms managed by them. Each room has been assigned to a tab containing a weekly overview. This works well for getting a general overview of what the different rooms are used for, but it makes it very difficult to e.g. find an available room, as one would have to browse through all the tabs one by one, until a suitable room is found. If one has requirements for specific equipment, it is nessesary to refer to a seperate paper (a Word document), listing the equipment available for each room.

\subsubsection*{Tabulex}
\label{subsub:tabulex}
Tabulex is the program used by the part of the Study Administration that allocates rooms for courses. In Tabulex, a user defines the relationship between classes, rooms, teachers and courses. Binding 1 of each together, you have an entity representing a class attending a specific course in a specific room held by a specific teacher. When all entities have been defined, Tabulex arranges them in a weekly schema, in such a way that no resources of different entities overlap, like making sure no teachers teach two course simoustanely and no rooms are used for two courses at the same time.\\

Tabulex is good at allocating general schemas, but it lack a lot of functionality relevant for ITU. When two courses overlap each other, which might happen if the distribution of resources makes it impossible for the algorithm to make a schema, Tabulex does not suggest any alternative solutions, or at least provide some information about the problematic entities in question. This makes it difficult, and very time consuming to come up with a proper alternative solution, as the user has to make a manual correction, which often results in another conflict. \\
Tabulex also does not keep track of equipment in each room. This is possible to solve through Tabulex though, as you can assign custom rules for the program to respect. As this is just simple rules, it might be hard to understand why these rules were created in the first place, if you were not the person responsible. This makes it hard for more than one person to use the application.\\
Finally, Tabulex does not keep track of the capacity of each rooms, which makes it hard to do a proper allocation, as you cannot put smaller classes in small rooms, and visa versa.

\label{sec:scope}
\section{Scope}
 Since we want to focus on the creation of a solid interface that solves the above problems, the implementation has been a second priority. We have divided the project into two main tasks, which together constitute to the main body of such an application. The success of the project depends on how well the interfaces for these two tasks have been designed.\\
In the following, we will describe the two main tasks and how they relate to each other. In general, we will work with all types of rooms except the offices, as they are handled differently. It would not make any sense to book an office, as they are assigned specifically to staff and students.

\subsection{Day to day booking}
\label{chap1:day_to_day_booking}
Our program should provide an interface to support booking of rooms (excluding offices) at the IT University. Users of this function might have very different requirements. Students might just need a room to work, while an employee might book a specific room in advance for a meeting. Others might need specific equipment to be available. Because of the many requirements, the program should not only provide the user with options, but also suggest solutions.\\
Since most people using our application will be students looking for a room without any specific requirements, the simple task of finding any available room should be easy accessible and quick to use. \\

The day to day booking should thus contain:

\begin{itemize}
	\item A general overview of the available auditoriums, meeting- and class rooms to book.
	\item An interface that quickly books a room, using a minimum of options. (\textbf{Quick booking})
	\item An interface for booking a room, selected by clicking on that specific room. (\textbf{Simple booking})
	\item An interface that presents the user with several options, such as the need for specific equipment, room capacity, time of day, etc. (\textbf{Advanced booking})
\end{itemize}

The application should:

\begin{itemize}
	\item Suggest \emph{a single room} to the user, when using the quickbook function. The room suggested, should be the \emph{optimal} room, in terms of efficient room allocation. This means that the application should suggest the smallest room.
	\item Suggest a list of rooms that fulfil the maximum number of requirements, given as input by the user through the selective booking function. If all requirements can be fulfilled, only the optimal rooms (see above) should be presented to the user.
\end{itemize}

To exemplify a working application, we will implement some parts of the day to day booking interface.

\subsection{Semester planning}
\label{chap1:semester_planning}
In addition to the day to day booking, we will design the user interface for a semester planning system. Planning a semester is in the case of ITU not just a matter of assignment. Each semester have more courses and day/time combinations than there is rooms, so the essence of it,is providing the tools to prioritize in an informed manner. At the point when a semester needs to be planned, most courses only have a start-week and an end-week, along with the number of enrolled students. Some courses can, eg. due to external teachers with busy schedules, have narrowed down their options, and we must of course including support for these scenarios.
Worth noting, and part of our considerations in the design, is that there will only be a few people using this part of the system, and these people will be easily accessible for training.

The semester planning should contain:

\begin{itemize}
	\item An interface to manage courses.
	\item An interface to manage tracks.
	\item An interface to handle any problems with planning that cannot be resolved automatically.
	\item An interface to view/review the current semester's allocation.
\end{itemize}

The application should:

\begin{itemize}
	\item Apply a constraint so that no teachers have to be attending two courses at the same or any overlapping time.
	\item Apply a constraint so that no courses are held in the same room at the same, or any overlapping time.
	\item Attempt to not put courses in meeting rooms, unless all class rooms are unavailable in all possible scenarios.
	\item Provide alternatives to assignments and allow for wishes/preferences.
	\item Provide suggestions for resolutions if no full semester plan is possible.
\end{itemize}


We will design the user interface for the semester planning, but not implement or add the required functionality. Doing the full scale implementation is not something we are able to do because of time constraints, while also not being crucial to prove our point of data presentation and usability.

\subsection{Relations}
\label{subsec:relations}
Even though the two tasks present very different functionality, the relationship between them needs to be defined, as they both occupy rooms. \\

The semester planning should have precedence at all times. This means that any day to day bookings interfering with the semester planning will be deleted. Because of this, it should also not be possible to book a room during the use of the semester planning interface. \\

In practice there might be exceptions, that we cannot predict. ITU might have a very important meeting that should not be deleted, or overwritten by the semester planning algorithm. These exceptions will be taken into consideration when implementing the application.

A day to day booking cannot be made, if it overlaps with a scheduled event (like a lecture) created through the event planning interface.\\

Not all users should have access to the semester planning. Only users defined as \emph{administrators} have access to this part of the system.

\subsection{Functionality out of scope}
Some functionality are not relevant to the functionality and interface of the two main tasks in our scope. The following list presents components that we discuss in this report, but do not implement.

\begin{itemize}
	\item A personal interface, listing all booking made by the user. Through this interface, it should be possible to review, edit and delete bookings. Each administrator of the system, should have access to \emph{all} booking, and should be able to approve or disapprove bookings.
	\item A communication protocol for sending and receiving messages. This is nessesary if e.g. a booking is deleted by an administrator through the semester planning interface.
	\item Some kind of restrictions to the day to day booking. It should e.g. not be possible for a user to book all available rooms at the same time.
\end{itemize}