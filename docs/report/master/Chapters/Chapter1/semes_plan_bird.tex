\subsection{Semester planning}
In addition to the day to day booking, we will design the user interface for a semester planning system. The interface should assist the administration in utilizing the rooms to the maximum potential by taking as many relevant factors in to consideration as possible. At the point when a semester needs to be planned, most courses only have a start-week and an end-week, along with the number of enrolled students. Some courses can, eg. due to external teachers with busy schedules, have narrowed down their options, and we must of course including support for these scenarios.
Worth noting, and part of our considerations in the design, is that there will only be a few people ever using this part of the system, and these people will be easily accessible for training.

The semester planning should contain:

\begin{itemize}
	\item An interface to manage courses
	\item An interface to manage the aforementioned tracks
	\item An interface to handle any problems with planning that cannot be resolved automatically
	\item An interface to view/review the current semester's allocation
\end{itemize}

The application should:

\begin{itemize}
	\item Attempt to not put courses in meeting rooms, unless all class rooms are unavailable in all possible scenarios
	\item Provide alternatives to assignments and allow for wishes/preferences
	\item Provide suggestions for resolutions if no full semester plan is possible
\end{itemize}