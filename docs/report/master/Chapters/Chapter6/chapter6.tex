\chapter{Conclusion}
In this project, we have developed a graphical user interface and a simple prototype, for a room management system for use at the IT University of Copenhagen.\\
We have divided the project into two main areas; a room booking system which can be used by all members of ITU, and a semester planning system, which should be used by the persons currently in charge of room allocation for the courses of a semester.\\
Our focus has been on the graphical development, because we think a good interface is of high importance in a system where overview is key. We have used the domain knowledge gathered throughout the project in our graphical user interface and prototype, to make the application as intuitive as possible to use for the target audience, while still providing support for the various tasks. \\

We have been using a iterative approach, which resulted in testable mockups early in the process. These mockups have been tested and redone through several rounds, with the help of several test subjects performing think aloud tests. We have tested the final version of our interface, and the results were generally satisfying for all purposes.\\

We have developed a working prototype, which demonstrates one way to book rooms using our system, and the general final concept of the user interface. The application has been implemented as a web-based solution, to make it easily accessible from anywhere. We have used the multi-paradigm programming language Scala in the development, and while it did give us some problems, it provided us with some important functionality for our project.\\

We have finally suggested the steps required to put our system to use, and discussed the impact our system would have on the domain it was built for.