\subsection{Functional requirements}
\label{subsec:functional_requirements}
To establish a foundation for the requirements of our system, we have considered the functionality our system must contain. These functional requirements only contains goals, and not \emph{how} they must be met, as we will adjust the way they take place based on usability factors.

The system must... \begin{enumerate}
\item distinguish users from eachother
\item store data and information persistently
\item support the user in performing the tasks specified in section \ref{sec:tasks}
\end{enumerate}

In regard to the first item, all students and faculty at the ITU already have unique logins, and it would be natural to reuse those for our system. Being a webbased system, it comes as no surprise that we wish the data to be stored persistantly so different computers at different times, without the loss of bookings or the like.
Naturally, the system has no use without actually fulfilling the tasks it has been designed to, hence the listing as a functional requirement.

Our functional requirements are fairly scarce compared to some systems, but this stems from the fact that the system, from an end-user perspective, is pretty simple. Our vision is simply to enforce this perspective, and provide the \emph{expected} functionality, without hindrance of high complexity.
