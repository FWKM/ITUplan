\chapter{Analysis}
\label{chap:analysis}
When unifying the functions of several departments and optimizing the overall process at the same time, a multitude of non-triviel problems will need answering. Through this chapter we will go through the discussions made for and against our choices.
\todo{Write a better introduction to this chapter}

\section{Target audience} % (fold)
\label{sec:target_audience}
Our application is implemented solely for ITU. It would have been possible to implement a general solution capable of adapting to many instutions, and while this is our general line of thought, a good UI needs to contain various elements specific to the domain in which it is used. This includes a map of the university for easy navigation and localization of rooms, and many application specific options. These features might not be nessesary for e.g. another university, which examplifies the difficulty in creating a general room-booking application without compromising the required functionality.\\

Our target audience can thus be divided into two groups:
\begin{itemize}
\item Student body at ITU
\item Employees at ITU
\end{itemize}

\subsection*{Student body}
The student body covers ages 18 - 40+ and with very different backgrounds. Based on the nature of the university, it is fair to assume that most students have atleast basic familiarity with IT and computer use in general.

\subsection*{Employees}
The computer proficiency of the employees can range from expert to novice, and no clear-cut distinctions can be made. We have, however, observed that the majority of the people our system would affect are using tools with some similarities, and are therefore assuming their proficiency to be above novice users.\\

Note that even though our day to day booking interface will mostly be used by students, we can make no assumptions that the target audience for this part of the application \emph{only} contains students. We need to make sure the day to day booking can be used by a very wide range of people, including also the employees.

\section{Requirements}
\label{sec:requirements}
% L�s s�ren la om Requirements
Since the ITU does not currently have a full system, we have put up initial requirements for such a system.
As described in the target audience \ref{sec:xx}, the system will be used by a wide variety of people, and there will be virtually no way to train them, thus putting a very high bar for user-friendliness. We have met this problem head-on by writing up all the tasks our system must support, along with quality requirements describing to what level our users must have succesfull uses. \cite{lauesen}

\subsection{Non-functional requirements} % (fold)
\label{subsec:non_functional_requirements}
Preparing for the creation of mockups following various usability tests, it is nessesary to set up a reference point to what we actually want to measure. This helps us to keep the following mockups and usability tests focused.\\
Usability quickly becomes very subjective, if you do not attempt to define it in more precise terms. We have used Soren Lausens five usability requirements \cite{lauesen}, as a measurement of the usability of our application.\\

In a perfect world, one might think that all of the 5 usability requirements would be met. However, this is rarely desired, as it does not emulate a clear focus of what should be prioritized in the application. Here follows a rundown of the usability factors and how they relate to our project. Note that the \textbf{fit for use} requirement has been left out. This is defined as:\\

\emph{Can the system support the tasks that the user has in real life?}\\

We do not think this is an actual usability requirement, as it is merely concerned with the functionality of the program.
\begin{itemize}
	\item \textbf{Task efficiency}\\
	Since the task of finding a room often will be done on the fly, on the way to school, or right before a meeting takes place, it is important that the application is responsive and gives the user a proper response to a booking query. It should give the user a good overview to support his decision, and suggest alternatives if his query cannot be met. In general, we have failed if the user finds it more effective to book a room the old fashioned way, by asking the Facility Management.
	\item \textbf{Understandability}\\
	The domain analysis suggested that the staff responsible for doing the semester planning had a large need to double-check that the important tasks performed by the system were done correctly. \ref{sec:xx}. In general, they had very little confidence in the system. Therefore, to avoid resistance towards change, it is important that the system communicates clearly with the user, during normal use, and expecially in error prone situations to avoid confusion.
	\item \textbf{Ease of learning}\\
	The set of people using our system is large and will change rapidly. It is therefore important that the system can be used "out of the box". If not, people will simply not use it, as it takes to much of their already limited time to get familiar with. Also, there is a close relation between ease of learning, and the effectiveness of the support the system offers. If the user does not understand how to use the system, they will indeed not be able to benefit from the support layer.
	\item \textbf{Subjective satisfaction}\\
	Some might argue that the subjective satisfaction is not very relevant in our case, because we're developing a tool for daily support and not a program that in any way should be amusing or entertaining. \\
However experts believe that users are more willing to accept the program and trust its efficiency if they find it visually appealing\cite{nielsen_metrics}. Apple is a good example of using design and pleasing aesthetics as major selling points. The iPad used this to brand itself, thus creating a need that in reality was not present. %Source?
\end{itemize}
%%%% REMEMBER TO ADD PERFORMANCE REQUIREMENTS AND STUFF %%%%

\subsection{Functional requirements}
\label{subsec:functional_requirements}
To establish a foundation for the requirements of our system, we have considered the functionality our system must contain. These functional requirements only contains high level goals, and not \emph{how} they must be met, as we will adjust the way they take place based on usability factors.

The system must... \begin{enumerate}
\item distinguish users from each other
\item store data and information persistently
\item support the user in performing the tasks specified in section \ref{sec:tasks}
\end{enumerate}

In regard to the first item, all students and faculty at the ITU already have unique logins, and it would be natural to reuse those for our system. Being a webbased system, it comes as no surprise that we wish the data to be stored persistantly so different computers at different times, without the loss of bookings or the like.
Naturally, the system has no use without actually fulfilling the tasks it has been designed to, hence the listing as a functional requirement.\\

Note that we have few functional requirements compared to some systems. This stems from the fact that the system, from an end-user perspective, is pretty simple. Our vision is simply to enforce this perspective, and provide the \emph{expected} functionality, without hindrance of high complexity and feature creep. \cite{porter}

\subsection{Quality Requirements}
\label{subsec:quality_requirements}
Based on the task list above, we can finally put forth the list of quality requirements we wish to meet in able to consider the level of usability acceptable.
\begin{enumerate}\itemsep1pt
\item Without instructions, 90\% of novice users must be able to complete T1.1 to T1.4 without variants.
\item Without instructions, 80\% of novice users must be able to complete T1.1 to T1.4 including variants.
\item With a maximum of 30 minutes instructions, 90\% of administration personel must be able to complete T2.1 to T2.5.
\item Without instructions, 50\% administration personel must be able to complete T2.1 to T2.5.
\end{enumerate}

\section{Tasks}
\label{sec:tasks}
The following tasks are the actions able to be performed within our system. The full list including subtasks covering all scenarios can be found in appendix \todo{ref til full task list}

\begin{itemize}\itemsep1pt
\item[\textbf{1.}] \textbf{Day-to-day booking}
\item[\textbf{T1.1}] Book room. May or may not be a specific room or a specific time/day
\item[\textbf{T1.2}] Check availability of room. May or may not be today.
\item[\textbf{T1.3}] Cancel booking. May or may not be before the booked timeslot.
\item[\textbf{T1.4}] Extend booking. May or may not be before the booked timeslot.
\\
\item[\textbf{2.}] \textbf{Semester planning}
\item[\textbf{T2.1}] Create course. May or may not have full information available at the point of creation.
\item[\textbf{T2.2}] Create track.
\item[\textbf{T2.3}] Edit course.
\item[\textbf{T2.4}] Edit track.
\item[\textbf{T2.5}] Plan semester. A full solution may or may not be possible
\end{itemize}




