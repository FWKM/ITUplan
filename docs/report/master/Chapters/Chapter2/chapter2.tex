\chapter{Analysis}
When unifying the functions of several departments and optimizing the overall process at the same time, a multitude of non-triviel problems will need answering. Through this chapter we will go through the discussions made for and against our choices.

\label{sec:scope}
\section{Scope}
Our approach to designing this system, has been primarily focused on usability and all that entails. As described in \ref{sec:current_situation}, the current situation holds multiple challenges. Tedious and de-centralized management of a single resource is both time consuming and highly error-prone.\\
We want to implement a system which fulfills two main tasks:
\begin{itemize}
	\item Helps the student to find a place to work
	\item Aids the staff in the allocation of rooms
\end{itemize}
A large part of fulfilling these tasks is the creation of an interface, which by itself maintains a high level of support. 
% Uddyb
Both of the above tasks can often be solved in multiple ways. A student query the database for a room with specific requirements, might lower his requirements if there is no results. On the other hand, he might decide to postphone  the activity and try again when the specific room is available.\\
In comparison, the room allocation problem can contain not only a few solutions (which a program might be able to suggest), but hundreds of ways to allocate courses. When something is impossible to solve, the user has to be able to tell the computer what to prioritize, or which solution to use.
A well thought out application succeeds in presenting this information to the user, and gives a proper overview so the best decisions can be taken. \\
Therefore, the far most important and interesting aspect of our application is the userinterface, in which we have choosen to put our focus. To exemplify a solid solution to this problem, and to keep the interface focused, we have narrowed our vision to the most used area: The day-to-day booking, and the semester planning (hence the tasks mentioned above). \\

The simple booking have been succesfully prototyped, whereas the semester planning have only been designed and developed theoretically, and not implemented. Doing the full scale implementation of the semester planning would be a possibility, but it is not crucial to prove our point of data presentation and usability. We have therefore chosen to perform the analytical work, and leave out large parts of the actual implementation.

\section{Requirements}
% L�s s�ren la om Requirements
Since the ITU does not currently have a full system, we have put up initial requirements for such a system. The requirements listed is the functionality that, when combined, provide a solution to the problem at hand.

\subsection{Use cases}
The booking of a room have four parameters; the week of the year, the day of the week, the time of day and of course, the room. With these four in mind, the searching, browsing and organising of rooms pose some relevant questions.
\\

\begin{center}
	\begin{tabular}{ | l | l | }
		\hline
		\multicolumn{2}{ | c | }{Legend} \\
		\hline
		W & Week \\
		D & Day \\
		T & Time \\
		R & Room \\
		\hline \hline
		Main case & High occurance \\
		Edge case & Rare occurance \\
		\hline
	\end{tabular}
\end{center}


\subsubsection*{With one parameter locked} 
Main cases: \\ \emph{
W - "I need a room sometime next week" \\
}\\
\noindent Edge cases: \\ \emph{
D - "I need a room on a wednesday sometime" \\
T - "I need a room at 12 o'clock sometime" \\
R - "I need Aud. 4 at some point" \\
}
\subsubsection*{With two parameters locked}
Main cases: \\ \emph{
WD - "I need a room at thursday next week" \\
WT - "I need a room at 9 o'clock next week" \\
WR - "I need Aud. 1 in 2 weeks" \\
DT - "I need a room mondays at 8 o'clock" (W free, so meaning a span of weeks needed here) \\
}\\
\noindent Edge cases: \\ \emph{
TR - "When is Aud. 4 available at 9 o'clock?" \\
}
\subsubsection*{With three parameters locked}
Main cases: \\ \emph{
WDT - "I need a room at 12 o'clock on friday next week" (Expecting this to be the most used search of them all) \\
WDR - "I need room 4a14 this friday" \\
}\\
\noindent Edge cases: \\ \emph{
WTR - "I need Aud. 3 at 9 o'clock next week sometime" \\
DTR - "I need room 2a09 on thurdays 11 o'clock" (Probably to be used along with a semi-locked W (span)) \\
}

\section{Design decisions}
\subsection{Platform}
Choosing a webbased solution over a desktop application was both a choice of interest and of usability. Accessibility is a crucial factor in usability, and by allowing end-users to access the system instead of having to go through the bottleneck of a single department, the workload is both distributed and handled immediately, as opposed to being dependant on an employee from said department to perform the whole process associated with locating and booking a suitable and available room.
\\
The drawbacks to \todo{finish this sentence}

\todo{How to implement, discussion of for and against our choices of platform, language, etc etc. Possibly split this in to several sections}

\subsection{Concurrent room selection}
The handling of concurrent users is always an issue in webbased services, and more specifically in our case it presents itself when multiple users wants to book a room at the same time. No solutions are perfect, and we've tried discovering the pros and cons of the solutions we considered.

\subsubsection*{Possible solutions}
These solutions are based on the the case where users \textbf{X} and \textbf{Y} are searching for rooms at the same time. They could obviously be extrapolated to cases where even more users are interacting. \\ \\
\emph{Make all rooms that fits \textbf{X}, unavaliable for \textbf{Y} while \textbf{X} is deciding on the which room to use.}
	\begin{itemize}
	\item Cons:
		\begin{itemize}
		\item If all rooms fit the search of \textbf{X}, no rooms will be available for \textbf{Y} 
		\end{itemize}
	\item Pros:
		\begin{itemize}
		\item Easy to implement.
		\item Extremely unlikely to encounter dead ends.
		\end{itemize}
	\end{itemize}
\emph{The screen which shows avaliable rooms, of user \textbf{Y} should update when user \textbf{X} makes a booking, to make sure \textbf{Y} doesnt hit a dead end.}
	\begin{itemize}
	\item Cons:
		\begin{itemize}
		\item Increased number of requests to the server.
		\item Dead ends possible.
		\item Ultimately, it is a tradeoff between probability of dead ends vs. server request-rate.
		\end{itemize}
	\item Pros:
		\begin{itemize}
		\item High usability and responsiveness, resulting in a pleasant experience for the user if no dead ends are hit. 
		\item More flexible; allows several users to have many choices at once.
		\end{itemize}
	\end{itemize}
