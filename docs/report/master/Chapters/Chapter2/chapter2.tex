\chapter{Analysis}
\label{chap:analysis}
When unifying the functions of several departments and optimizing the overall process at the same time, a multitude of non-triviel problems will need answering. Through this chapter we will go through the discussions made for and against our choices.

\section{Domain analysis} % (fold)
\label{sec:domain_analysis}

% section domain_analysis (end)

\section{Requirements}
\label{sec:requirements}
% L�s s�ren la om Requirements
Since the ITU does not currently have a full system, we have put up initial requirements for such a system.
As described in the target audience \todo{ ref}, the system will be used by a wide variety of people, and there will be virtually no way to train them, thus putting a very high bar for user-friendliness. We have met this problem head-on by writing up all the tasks our system must support, along with quality requirements describing to what level our users must have succesfull uses. \cite{lauesen}

\todo{This section should conclude with a initial list of required tasks}

\subsection{Non-functional requirements} % (fold)
\label{subsec:non_functional_requirements}

% subsection non_functional_requirements (end)

\subsection{Functional requirements} % (fold)
\label{sub:functional_requirements}

% subsection functional_requirements (end)

\subsection{Quality Requirements}
\label{subsec:quality_requirements}
Based on the task list above, we can finally put forth the list of quality requirements we wish to meet in able to consider the level of usability acceptable.
\begin{enumerate}\itemsep1pt
\item Without instructions, 90\% of novice users must be able to complete T1.1 to T1.4 without variants.
\item Without instructions, 80\% of novice users must be able to complete T1.1 to T1.4 including variants.
\item With a maximum of 30 minutes instructions, 90\% of administration personel must be able to complete T2.1 to T2.5.
\item Without instructions, 50\% administration personel must be able to complete T2.1 to T2.5.
\end{enumerate}

\section{Tasks}
\label{sec:tasks}
The following tasks are the actions able to be performed within our system. The full list including subtasks covering all scenarios can be found in appendix \todo{ref til full task list}

\begin{itemize}\itemsep1pt
\item[\textbf{1.}] \textbf{Day-to-day booking}
\item[\textbf{T1.1}] Book room. May or may not be a specific room or a specific time/day
\item[\textbf{T1.2}] Check availability of room. May or may not be today.
\item[\textbf{T1.3}] Cancel booking. May or may not be before the booked timeslot.
\item[\textbf{T1.4}] Extend booking. May or may not be before the booked timeslot.
\\
\item[\textbf{2.}] \textbf{School Administration}
\item[\textbf{T2.1}] Create course. May or may not have full information available at the point of creation.
\item[\textbf{T2.2}] Create track.
\item[\textbf{T2.3}] Edit course.
\item[\textbf{T2.4}] Edit track.
\item[\textbf{T2.5}] Plan semester. A full solution may or may not be possible
\end{itemize}

\section{Usability factors} % (fold)
\label{sec:usability_criteria}

% section usability_criteria (end)


