\section{Usability Tests}
\label{app:tests}
\subsection{Day to day booking}
\subsubsection{Round 1}
The following scenarios were presented to the test subjects:
\begin{enumerate}
\item You and your group have met, and intend to conduct group-work without being disturbed.
\item Your group has finished work today, and decided to reconvene tomorrow. You have been assigned to arrange for a place to work.
\item Your group is done with the assignment and wish to practice your powerpoint presentation using a projector in the near future.
\item While working in <room which was just found> you have been interrupted by another group asking when the room is next available.
\item Add-on: They refuse the first time you suggested as they have gone home by then/don't work on that day/have vacation that week.
\end{enumerate}

Test subject 1, ITU student1:
\begin{enumerate}
\item Presses 'Find me a specific room'\\
Skims through, looking for an easier option \\
Fills out search options partly, then realizes it's too complicated and clicks cancel \\
Hesitantly presses 'Find me a room' \\
Picks start and end time \\
Presses book\\
\textbf{Result: Task failure due to overcomplication - Button labels are unclear}
\item Presses 'Find me a specific room' \\
Spots calendar just after clicking, clicks cancel and selects tomorrow on the calendar \\
Clicks 'find me a room' \\
fills in and accepts \\
\textbf{Result: Minor problem - drawn towards 'specific' instead of using calendar when other dates are involved}
\item Confidently presses 'find me a specific room' \\
Checks off 'projector' and todays date, leaves rest default \\
\textbf{Result: Pass - no remarks.}
\item Initially puzzled. Asks: ``Can I click the map?'', clicks before evaluator confirms \\
Clicks the room he was granted from search \\
Uses finger to trace timeline from left to right \\
Provides multiple options: Later today and tomorrow (with specific time intervals) \\
\textbf{Result: Pass - Note: doubt whether map was clickable. }
\item Add-on excluded as test subject already provided multiple answers to 4.
\end{enumerate}

Test subject 2, tech savvy outsider:
\begin{enumerate}
\item Skims interface and hesitates. Clicks 'find me a room' \\
Fills out start and end time \\
Doesn't notice header showing which room was chosen \\
clicks a random room on map \\
Realizes mistake and clicks cancel \\
Clicks 'find me a room', fills out again and presses book \\
\textbf{Result: Task failure due to overcomplication - Didn't notice that a room was assigned - Header hidden by clutter}
\item Presses 'find me a room' \\
Clicks 'Another day' and selects tomorrow \\
Fills in time and presses book \\
\textbf{Result: Pass - note: chose 'another day' instead of calendar, both are valid}
\item Clicks 'find me a room' \\
Fills in time and accepts \\
Evaluator informs that the room does not have the required tools (projector) \\
Clicks 'specific', checks off projector, searches and books. \\
\textbf{Result: Medium problem - Simplifies problem leading to potential task failure, but lowered severity to medium as question formulation might have been the problem, and task was solved flawlessly once problem was understood}
\item Clicks the room on map \\
Informs that room is available from 13.00 today \\
\textbf{Result: Pass - no remarks.
\item Evaluator declines 13.00 today and requests next week \\
Clicks calendar \\
Informs that monday 9.00 is free \\
\textbf{Result: Pass - no remarks}

\subsubsection{Round 2}
The following scenarios were presented to the test subjects:
\begin{enumerate}
\item You and your group have met, and intend to conduct group-work without being disturbed.
\item Your group has finished work today, and decided to reconvene tomorrow. You have been assigned to arrange for a place to work.
\item Your group is done with the assignment and wish to practice your powerpoint presentation using a projector in the near future.
\item While working in <room which was just found> you have been interrupted by another group asking when the room is next available.
\item Add-on: They refuse the first time you suggested as they have gone home by then/don't work on that day/have vacation that week.
\item Group member calls in sick, and work tomorrow is cancelled. (cancel reservation)
\item You have been tasked with gathering all the first-year students for a surprise announcement in Auditorium 1. Date is not set.
\end{enumerate}

Test subject 3, ITU student2:
\begin{enumerate}
\item Clicks a meeting room on map \\
Fills in time and books\\
\textbf{Result: Minor problem - Solves task by using knowledge of ITU}
\item Clicks calendar tomorrow \\
Clicks find room and accepts \\
\textbf{Result: Pass - no remarks}
\item Clicks specific \\
Checks off projector and books \\
\textbf{Result: Pass - no remarks}
\item Clicks room on map \\
Dialogue does not offer a way to view tomorrow, subject is puzzled \\
Cancels, and begins reading, realisizes calendar can be used \\
clicks calendar, clicks room, provides answer \\
\textbf{Result: Medium problem - functionality unintuitive/lacking}
\item Browses days by calendar \\
\textbf{Result: Pass}
\item Starts reading interface \\
Does not notice top panel until almost giving up \\
Selects booking, cancels booking \\
\textbf{Result: Annoying - Did not spot until having read through whole interface}
\item Clicks auditorium 1 \\
Begins browsing with calendar and decides on next tuesday \\
\textbf{Result: Pass - no remarks}
\end{enumerate}

Test subject 4, ''average'' outsider1:
\begin{enumerate}
\item Clicks 'find room' \\
Accepts \\
\textbf{Result: Pass - no remarks}
\item Clicks 'find room' \\
Overlay covers calendar, can't click \\
clicks cancels \\
<note: forgets about calendar it seems> \\
Click 'find specific' \\
Fills in date and time for tomorrow \\
\textbf{Result: Annoying - hits dead end with 'find room' + calendar combination}
\item Clicks 'specific' \\
Fills in, checks 'projector' \\
\textbf{Result: Pass}
\item Clicks 'specific' \\
Cancels \\
Clicks 'find room' \\
Cancels \\
Starts reading the interface \\
Clicks room \\
Cancel \\
Gives up \\
\textbf{Result: Task failure - Subject not able to complete assigned task (calendar + room click combo would have solved)
\item Add-on skipped due to task failure
\item Instantly starts checking top menu \\
Clicks reservations \\
cancels booking \\
\textbf{Result: Pass - subject is used to convention about menu bar}
\item Clicks aud1 \\
clicks calendar \\
\textbf{Result: pass - 'date' must have been keyword that made him think of the calendar again. }
\end{enumerate}

\subsubsection{Round 3}
The following scenarios were presented to the test subjects:
\begin{enumerate}
\item You and your group have met, and intend to conduct group-work without being disturbed.
\item Your group has finished work today, and decided to reconvene tomorrow. You have been assigned to arrange for a place to work.
\item Your group is done with the assignment and wish to practice your powerpoint presentation using a projector in the near future.
\item While working in <room which was just found> you have been interrupted by another group asking when the room is next available.
\item Group member calls in sick, and work tomorrow is cancelled. (cancel reservation)
\item You have been tasked with gathering all the first-year students for a surprise announcement in Auditorium 1. Date is not set.
\end{enumerate}

Test subject 5, ''average'' outsider2:
\begin{itemize}
\item Presses 'instant booking' \\
Accepts offered room \\
\textbf{Result: Pass - no remarks}
\item Clicks tomorrow on calendar \\
Presses 'instant booking' and accepts \\
\textbf{Result: Pass - seems to have grasped how 'instant booking' works right from the get go}
\item Clicks 'advanced' \\
Fills in, and selects projector \\
\textbf{Result: Pass - no remarks
\item Clicks assigned room on map \\
Uses week overview to provide options \\
\textbf{Result: Pass - test subject comments that the week overview is very nice}
\item Clicks reservations in top menu \\
Selects bookign and cancels it \\
\textbf{Result: Pass}
\item Click aud1 \\
Suggests day next week \\
Add-on: Evaluator declines and says that this month won't work \\
Browses a month forward on calendar and clicks a random day \\
Suggest new day/time \\
\textbf{Result: Pass + add-on - Test subject performed extrordinarily well.}
\end{itemize}


Test subject 6, novice outsider
\begin{itemize}
\item Presses 'instant booking'
Accepts \\
\textbf{Result: Pass - no remarks}
\item Hesitates, clicks calendar tomorrow \\
clicks instant and accepts \\
\textbf{Result: Pass - hesitation for a few seconds, dismissed as unfamiliarity with test format}
\item Clicks 'advanced' \\
Fills out everything, selects projector \\
selects and books \\
\textbf{Result: Pass - subject may not have noticed that most things can be left out}
\item Clicks 'reservations' \\
Informs when his booking ends \\
Evaluator declines - today is no good \\
Returns, click 'instant' \\
quickly realisizes mistake, cancels \\
clicks room on map \\
provides answer from overview \\
\textbf{Result: Minor problem - Used reservation screen initially}
\item Clicks 'reservations' \\
cancels booking \\
\textbf{Result: Pass - no remarks}
\item Clicks advanced \\
fills out, checks off 'auditorium' \\
finds aud1 and books \\
Evaluator informs that today is no good \\
Subject is puzzled - clicks 'instant', cancels \\
Clicks calendar then 'instant', cancels \\
Remembers map, clicks aud1 \\
suggests new date \\
\textbf{Result: Task failure - Didn't realise map was usable for this, although it had been clicked before. Almost solved by muddling through}
\end{itemize}

\subsection{Semester planning}

\subsubsection{Round 1}
\begin{verbatim}
Round 1: Semester planning 

---------------------------------
Purpose of test
---------------------------------
The purpose of this test is to find out whether our UI fills the following criteria:

1. Contains enough functionality to support the full task of assigning rooms to courses for a semester
2. Is designed so that the target-user will be able to fulfill the task with little or no instructions


More specifically, the points we need to address is:

3. Is our attempt at 'clash resolution' satisfactory?
4. Is the 'track management' efficient enough for the problem at hand?
5. Will the target-user protest against the manual entering of courses etc, even though there is possibilty for re-use over the course of years?


---------------------------------
Tasks to perform
---------------------------------
The following tasks are what we need the user to perform:

---------- Initiation tasks ----------
I1. Add a course
I2. Add a course that cannot be moved by auto-assignment
I3. Remove the restriction from the course added by T2/T1
I4. De-active a course that is already created
I5. Create a new track
I6. Edit a track (add a course and remove another one eg.)
---------- Main tasks -----------
M1. Plan a semester (no clash happens)
M2. Plan a semester (clash happens)
 M2.1. Resolve the clash automatically
 M2.2. Resolve the clash manually
---------- Rare task -----------
R1. Clear the room assignment to make next semester ready



---------------------------------
Test course
---------------------------------
I1
I2
I4
I5
I6
M2
M2.1
M2.2
I3
M1
R1


---------------------------------
Lead-ins
---------------------------------
I1 - Add a course

We already added a few courses to the system, but forgot one. Please add BTES
[ TEST ]|[ BTES ]
[ Week 30 ]|[ Week 49 ]
[ 4 hours ]|[  ]
[ 42 ]|[  ]
[  ]|[  ]

Correct solution:
-> 'Add new course' tab
-> fill in
-> 'Save'


I2 - Add a course that cannot be moved by auto-assignment

Another course has been forgot, this one needs to be in AUD4 for some wierd reason
[ AUD ]|[ BAUD ]
[ Week 30 ]|[ Week 49 ]
[ 4 hours ]|[  ]
[ 42 ]|[  ]
[ AUD4 ]|[ X ]

Correct solution:
-> 'Add new course' tab
-> fill in
-> 'Save'


I4 - De-active a course that is already created

Oops, turns out that BTES didn't get enough students this semester and have been put on hold this semester. Lets not delete it, but make sure it daasn't get a room.

Correct solution:
-> 'Edit course' tab
-> Click BTES
-> Click Edit
-> Check 'Inactive box'
-> 'Save'


I5 - Create a new track

BPRD and BSDB are on the same track, and should therefore not be planned at the same time. Lets add the track.

Correct solution:
-> 'Manage tracks' tab
-> fill in name
-> Select courses using the arrows
-> 'Save'



I6 - Edit a track (add a course and remove another one eg.)

DMD has taken a new direction and now needs more programming. Thus, BPRD have been added to the first semester. Fix it!

Correct solution:
-> 'Manage tracks' tab
-> Click 'DMD - sem 1' in the list
-> Click 'Edit
-> Select courses using the arrows
-> 'Save'


M2 - Plan a semester (clash happens)
Alright, we're all set. Lets assign the rooms!

Correct solution:
-> Any tab
-> Click 'Auto-assign'
-> BAM CLASH

Depending on initial reaction, go to either m2.1 or m2.2

M2.1 - Resolve the clash automatically

Fix it!

Correct solution:
-> Accept one of the solutions

M2.2 - Resolve the clash manually

Fix it!

Correct solution:
-> Switch tab to edit course
-> Edit shit until it works
-> try again



I3 - Remove the restriction from the course added by I2/I1

The course BAUD have lost it's privileges and can no longer claim AUD4

Correct solution:
-> 'Edit course' tab
-> Click BAUD
-> Click Edit
-> Check 'required off' / Clear room box
-> 'Save'




M1 - Plan a semester (no clash happens)

We've fixed all the stuff that could clash, lets do it!

Correct solution:
-> 'Auto-assign'
-> ???
-> Profit



R1 - Clear the room assignment to make next semester ready

This semester was a blast, lets try again.

Correct solution:
-> 'Room assignment' tab
-> Click the danger button
-> accept the insane warning (possible text input)

RESULTS OF USABILITY TEST

USER	Malene de Bruin
	Studieadministration

	Malene is head of semester planning. At the start of each semester, she plans and coordinates the rooms for the various courses, using tabulex and a homemade spreadsheet.
	She is thus the exact person that will be using the semester planning part of our application.


---------- Initiation tasks ----------
I1. Add a course
Skriver kursus, trykker paa start week. Vaelger 30. Er i tvivl om antal uger (?). Vaelger duration. "N����H, gaar altid ned af og ikke hen af.
"Starttidspunkt, ville jeg umiddelbart lade staa blankt. Ville det vaere tydeligt at systemet vaelger det for mig. Tror jeg ville vaelge klokken 8, uden at vide hvorfor!".
"Weekdays, not selected. Der er ikke noget room-request. Ville proeve at klikke paa room required, da der er en checkbox". Forstaar meningen.


I2. Add a course that cannot be moved by auto-assignment
Skriver navnet og kode, vaelger det samme som foer. "Jeg er meget i tvivl om optional, og hvad der sker naar jeg ikke vaelger det". Vaelger required og gemmer.


I3. Remove the restriction from the course added by T2/T1
Trykker paa edit course. Vaelger kurset fra listen, trykker edit. Fjerner required -> save.


I4. De-active a course that is already created
Trykker paa edit courses. "Ahhh, jeg skal fremkalde BTES". Vaelger i listen, og trykker edit. Vaelger inactive, save.


I5. Create a new track
Klikker paa manage track. Flytter dem til added, og skriver et navn. Trykker paa save.


I6. Edit a track (add a course and remove another one eg.)
Markerer track, trykker paa edit. Markere og flytter, save.


---------- Main tasks -----------
M1. Plan a semester (no clash happens)


M2. Plan a semester (clash happens)
"Det jeg ikke kan se er hvad problemet er. Ellers ret god loesning at den foreslaar alternativer. Hvad saa hvis man ikke accepterer?. Jeg vil gerne selv goere det manuelt, men det kan jeg ikke. AHHHH, det er da klart". (Efter Bird har vist hende firkanten med tekst).

 M2.1. Resolve the clash automatically
 M2.2. Resolve the clash manually

---------- Rare task -----------
R1. Clear the room assignment to make next semester ready
"Ville have det daarligt med at slette alle kurserne een for een. Ville hellere goere dem inaktive; ville foeles bedre hvis de ligger der, og ikke bliver slettet".
Finder clear plan efter Bird har uddybet. Finder knappen. "Det er meget rarere at knappen er der, clearer og ikke sletter".
\end{verbatim}
\subsubsection{Round 2}
Round 2 followed immediately after round 1, as changes were made to mockups as we moved along by the person not evaluating the test, to conserve the test subjects time.
