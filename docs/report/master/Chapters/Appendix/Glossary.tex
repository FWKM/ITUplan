\clearpage
\pagebreak
\section{Glossary}
\label{sec:glossary}
\begin{itemize}
\item \textbf{Advanced booking}\\
The process of using our advanced booking interface to book a room, specifying requirements which the room should fulfill. 
	
\item \textbf{Booking}\\ 
The entity that represents a room reserved by a user, at a specific point in time.

\item \textbf{Clash}\\
When one or more constraints conflict, making it impossible for the semester planning algorithm to create a full semester plan.

\item \textbf{Constraint}\\
A rule that always should be satisfied by the semester planning algorithm. A constraint can be built into the algorithm, or be passed as an input. An example of a built constraint, is the rule that no courses should be held in the same room at the same time.

\item \textbf{Client}\\
An application or system that accesses a remote service on another computer. In our case, the clients are the browsers used by our users, rendering the html pages of our web application, and showing the information gathered from the server.

\item \textbf{Day to day booking}\\
The part of our application and interface concerned with the booking functionality, where members of ITU can reserve a room for a period of 4 hours.

\item \textbf{Evaluator}\\
The person executing a think aloud test with a test user.

\item \textbf{Gestalts}\\
Design laws, describing how we perceive relationships between geometric elements.

\item \textbf{Javascript}\\
A scripting language used to manipulate the client side of our application.
	
\item \textbf{Lift}\\
An Open Source web application for Lift, used for porting a Scala implemention to an actual working web application.

\item \textbf{Mockup}\\
A quickly created layout and interface of a single part of an application, used to quickly test a proposed design.

\item \textbf{Prototype}\\
A partly implemented application, used to test functionality and/or design.


\item \textbf{Quick booking}\\
The simplest way of booking a room with our application. Day to day booking is the process of clicking the "Quick Booking" button, which books any room at ITU, which is available at the current time for a period of 4 hours.

\item \textbf{SBT}\\
Simple Build Tool. Used to build and compile a Lift application. SBT comes with a local web server which can be used to run a fully functional local copy of a lift application.	


\item \textbf{Scala}\\
The server-side programming language used through out the project. Scala is based on the Java Virtual Machine, making it object oriented, but with certain features taken from functional programming.

\item \textbf{Semester planning}\\
The part of our application concerned with the process of creating a plan for a semester; creating courses and assigning rooms to them.

\item \textbf{Server}\\
The server provides the service called by the client. In our application, the server handles the bookings and users of our application.

\item \textbf{Session}\\
One session starts when a user succesfully signs in to our application. It ends when they log out, or close the browser.

\item \textbf{Simple booking}\\
Covers the process of booking a room using the map on the frontpage, then selecting a day of the week and a timespan in which the room should be reserved.

\item \textbf{Tacit knowledge}\\
Knowledge that is difficult to transfer to another person by means of writing or verbal communication. In our project, the persons residing in the domain have tacit knowledge, which we should take into consideration when designing the interfaces.

\item \textbf{Test subject/user}\\
The person that performs the tests in a think aloud test, controlled by an evaluator. 

\item \textbf{Think aloud test}\\
A method of usability testing. Involves test users performing specific tasks, while explaining what and why they do what they do.

\item \textbf{Usability / User-friendly}\\
Usability is the ease of use and learnability of human made objects. Objects with high usability are easy to use, easy to learn, intuitive and user friendly.

\item \textbf{User}\\
In our application, a user is someone cabable of using our applications. This means all persons registrered in the userbase of ITU.

\item \textbf{User experience}\\
A good user experience for a user, means that the user liked a specific product or service. User experience is subjective, and there is not necessarily a connection between user experience and usability.

\item \textbf{Wireframe}\\
A drawn mockup, with focus on the data that the particular interface should include. Used to evaluate a design idea with actual data.

\end{itemize}