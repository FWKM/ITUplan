When designing a prototype, there is no systematic way of deciding how many screens are needed, and what they need to contain \cite{lauesen}.


Based on the user and task analysis conducted in chapter \ref{chap:analysis}



\subsection{Portability}
It could be interesting to expand the system to allow for use by other institutions. The user interface itself, bar the map, is already reusable, but in able for other instutions with other policies and work distribution, some changes would be in order.
For example:
\begin{itemize}
\item Support for multiple buildings
\item Extensive administration panel for fine-tuning permissions, setting default values and managing rooms
\item Additional parameters for semester planning in case tracks are not sufficient
\end{itemize}

Naturally, ITU could also benefit from these, but they are not immidiately vital.

\subsection{Expanding for the ITU}
A future expansion specific to ITU, is the ability to manage the offices we have excluded from our scope. In theory, it could function a lot like the semester planning, but would obviously be less complex, due to the fact that offices are booked for a longer period, several people can fit into them and there are currently enough space for all the people requiring it.

