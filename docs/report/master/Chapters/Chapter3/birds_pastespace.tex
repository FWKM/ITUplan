When designing a prototype, there is no systematic way of deciding how many screens are needed, and what they need to contain \cite{lauesen}.


Based on the user and task analysis conducted in chapter \ref{chap:analysis}



\section{Impact}
\label{sec:impact}
The following is an attempt to estimate what impact a fully implemented version of our system would have for the relevant stakeholders.

\subsection{Launch}
Before a fully implemented system could be launched, policies and decisions should be made for the issues raised in section \ref{sec:domain_challenges}.
With any kind of changes to current procedures in an institution of this size, there will always be a transistion period where things are not being used properly, not being used at all. We hope that the intuitive and user friendly approach we have taken, will help ease this particular part of the launch.

\subsection{Facilities Management}
As previously stated, FM handles the day to day booking manually, by manning a desk and browsing/updating a binder.
With our system it will depends on whether or not they choose the approval approach mentioned in section \ref{subsec:approval}.
In case of using approval, their work would be faster, but in the same approximate amount.
If they instead choose to follow our recommendations and allow users to make their own bookings, in limited number, and have the power to delete them, their time managing rooms could be greatly reduced.

\subsection{Study Administration}
This is the single greatest benefactor of a succesful implementation, as the current semester is highly time consuming, and quite troublesome. With our system, the need for tabulex would be gone, along with the manual work to fill the gaps that tabulex cannot do automatically. Through our discussion with the representative from the Study Administration, she mentioned that getting the overview of whether a not a track was fully assigned was the real problem. Our proposed design suggest a track-based approach with easy overview on how the schedule and assignments for a single track looks. The automated suggestions for clash-handling would also enable them to make educated decisions on what courses need to be moved and which rooms to be reassigned. Naturally, our system would not be able to solve all the problems encountered in such a complex assignment, but \emph{any} suggestion is more than what they have currently.

\subsection{Students, faculty and staff}
For the daily visitors to ITU, finding a meeting room or workspace would be greatly simplified, as the need to wait in line at the information desk would be eliminated. Additionally, rooms can be reserved from home, so no one would have the dissappointment of showing up, only to discover that no room is available.
