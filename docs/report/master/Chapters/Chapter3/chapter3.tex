\chapter{User interface}
\todo{general introduction to this section, and why it warrents a whole chapter}

\section{Challenges}
\todo{What challenges must our user interface overcome?}
\\
% info found in: 'concurrent_room_selection-11-02-03', 'meeting_study-administration_11-03-16' and 'courses_spanning_multiple_days'

\section{Use cases}
The booking of a room have four parameters; the week of the year, the day of the week, the time of day and of course, the room. With these four in mind, the searching, browsing and organising of rooms pose some relevant questions.
\\
To assist us in the process of deciding which parameters in a search or overview should determine the presentation, we postulate the following use cases based on number of 'locked' parameters.
\begin{center}
	\begin{tabular}{ | l | l | }
		\hline
		\multicolumn{2}{ | c | }{Legend} \\
		\hline
		W & Week \\
		D & Day \\
		T & Time \\
		R & Room \\
		\hline \hline
		Main case & High occurance \\
		Edge case & Rare occurance \\
		\hline
	\end{tabular}
\end{center}


\subsection*{With one parameter locked} 
Main cases: \\ \emph{
W - "I need a room sometime next week" \\
}\\
\noindent Edge cases: \\ \emph{
D - "I need a room on a wednesday sometime" \\
T - "I need a room at 12 o'clock sometime" \\
R - "I need Aud. 4 at some point" \\
}
\subsection*{With two parameters locked}
Main cases: \\ \emph{
WD - "I need a room at thursday next week" \\
WT - "I need a room at 9 o'clock next week" \\
WR - "I need Aud. 1 in 2 weeks" \\
DT - "I need a room mondays at 8 o'clock" (W free, so meaning a span of weeks needed here) \\
}\\
\noindent Edge cases: \\ \emph{
TR - "When is Aud. 4 available at 9 o'clock?" \\
}
\subsection*{With three parameters locked}
Main cases: \\ \emph{
WDT - "I need a room at 12 o'clock on friday next week" (Expecting this to be the most used search of them all) \\
WDR - "I need room 4a14 this friday" \\
}\\
\noindent Edge cases: \\ \emph{
WTR - "I need Aud. 3 at 9 o'clock next week sometime" \\
DTR - "I need room 2a09 on thurdays 11 o'clock" (Probably to be used along with a semi-locked W (span)) \\
}

\section{Mockups}
\todo{Add all/some of the pictures from our mockups. Currently located on github.
Explan the iterations.}

\section{Usability}
\todo{Usability discussion and documentation for findings from our tests}
\\
% info found in: 'test _mockup1' and 'Courseplan_test_master'
\subsection{Test af mockup1}

Tasks to perform\\
1 - simple: "Book a room today" \\
2 - simple: "Book a room tomorrow" \\
3 - special case: room not available current day: "Book room x for use as soon as possible" \\
- ... \\


Questions to ask \\
- "cancel" or "back" ? : "what do you think this button does?" \\
- Calendar - is it obvious? \\


Lead-in \\
1.
You and your group have met, and need to find a room to conduct group-work.\\

2.
Your group has finished work today, and decided to reconvene tomorrow. You have been assigned to arrange for a place to work tomorrow.\\

3.
Your group will be having it's exam in auditorium 3, and you wish to prepare for it in the same setting as the actual exam. \\


11 - 11.20:	Jakob 27 DMD 2. semester\\



1. Find a room\\
Trykker på Find me a Room \- siger trykker vel her.
Vælger tid fra og til.\\


2. Rum i morgen\\
Trykker på find me a specific room. Finder hurtigt ud af at han nok skulle have valgt vha. kalenderen.\\


3. Eksamen i auditorie3 en anden dag.\\
Trykker på auditorie3 på kortet. Trykker på another day. \\


4. Find rum næste fredag.\\
Find me a specific room. Vælger "projektor", og er nice. Spørger om man skal vælge tid her.\\

5. Find et lokale til mere end 50 mennesker på en onsdag.\\
Find me a room, cancel, find me a specific room. Task failure. Går tilbage til kortet. Trykke på et af auditorierne. Siger at han nok i stedet ville trykke på kalenderen. \\


6. Komsammen for førsteårsstuderende.\\
Trykker på aud1 på kortet. Bruger kalenderen til at vælge datoer, for at se ledigheden.\\




Jon, 23, SWU 6. semester\\

1.
Find me a room, OK. \\


2.
Bruger kalendervælgeren til at finde datoenm find me a room, OK.\\


3.
Vælger dato gennem kalenderen, find me a specific room, OK.\\


4. 
Som 3. \\


5.
Find me a room. Cancel. Trykker på w på kalenderen. Task failure, specific room. Vælger "auditorie" og ikke capacity. OK. \\


6.
Trykker på aud1 på kortet. Trykker på datoen, trykker på start-tid og slut-tid. Trykker på book.

\section{Design process}
\todo{explain the process of the graphical design}

\section{Proposed design}
\todo{Explain and illustrate the proposed design}
